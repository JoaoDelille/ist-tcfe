\section{Theoretical Analysis}
\label{sec:analysis}

In this theoretical analysis, used the circuit's transfer function to compute the predicted values for the output frequency ($V_{out}$) in function of the input voltage ($V_{in}$) of 10V. Furthermore, the input and output impedances were calculated.\\
The transfer function is the following: \[\frac{R_1C_1s}{1+R_1C_1s} \cdot (1+\frac{R_3}{R_4}) \cdot \frac{1}{1+R_2C_2s}\] \\

The bandwith is given by the inteval from: $\frac{1}{R_1C_1}$ to $\frac{1}{R_2C_2}$; the corresponding central frequency is \ $\sqrt{{\frac{1}{R_1C_1} \cdot \frac{1}{R_2C_2}}}$\\

The values obtained for the bandwith and central frequency were:

\FloatBarrier
\begin{table}[h]
  \centering
  \begin{tabular}{|c|c|}
    \hline    
    \input{resources/f_c}
    \hline
  \end{tabular}
  \caption{Bandwith and Central Fequency}
  \label{tab:Octave_cent}
\end{table}
\FloatBarrier

The corresponding output voltages for an input voltage of 10V were:
\FloatBarrier
\begin{table}[h]
  \centering
  \begin{tabular}{|c|c|}
    \hline    
    \input{resources/Vo_oc.tex}
    \hline
  \end{tabular}
  \caption{Bandwith and Central Fequency $V_{out}$}
  \label{tab:Octave_cent}
\end{table}

\FloatBarrier

The input and output impedances were:
\FloatBarrier
\begin{table}[h]
  \centering
  \begin{tabular}{|c|c|c|}
    \hline    
    \input{resources/Z.tex}
    \hline
  \end{tabular}
  \caption{Input and output impedances}
  \label{tab:Octave_cent}
\end{table}
\FloatBarrier

Using the transfer function T(s), it was possible to calculate the $V_{out}$ for a range of frequencies.

\begin{figure} [!htb] 
  \minipage{0.9\textwidth}
  \includegraphics[width=\linewidth]{resources/gain.png}
  \caption{$V_{out}$ amplitude in function of frequency} 
  \label{fig:theoplots}
  \endminipage\hfill
\end{figure}

\FloatBarrier

The frequency response is:

\begin{figure} [!htb] 
  \minipage{0.9\textwidth}
  \includegraphics[width=\linewidth]{resources/T(s).png}
  \caption{Gain} 
  \label{fig:theoplots}
  \endminipage\hfill
\end{figure}

\FloatBarrier

%\begin{figure} [!htb] 
 % \minipage{0.9\textwidth}
  %\includegraphics[width=\linewidth]{resources/phase.png}
  %\caption{Phase} 
  %\label{fig:theoplots}
  %\endminipage\hfill
%\end{figure}

\FloatBarrier

\begin{figure} [!htb] 
  \minipage{0.9\textwidth}
  \includegraphics[width=\linewidth]{resources/phase_deg.png}
  \caption{Phase} 
  \label{fig:theoplots}
  \endminipage\hfill
\end{figure}

\FloatBarrier




