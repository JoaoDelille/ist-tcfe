\section{Comparisons}
\label{sec:comparsisons}

In this section, a comparision between the ngspice and octave results is made. Moreover, we delve into the reasons regarding the diffences and similarities observed\\
Starting with the frequency response, the gain and phase are presented bellow.
Gain:
\begin{figure} [!htb] 
  \minipage{0.9\textwidth}
  \includegraphics[width=\linewidth]{}
  \caption{NGspice voltage gain}
  \label{fig:theoplots}
  \endminipage\hfill
\end{figure}

\begin{figure} [!htb] 
  \minipage{0.9\textwidth}
  \includegraphics[width=\linewidth]{}
  \caption{Octave voltage gain}
  \label{fig:theoplots}
  \endminipage\hfill
\end{figure}

Phase:
\begin{figure} [!htb] 
  \minipage{0.9\textwidth}
  \includegraphics[width=\linewidth]{}
  \caption{NGspice voltage phase}
  \label{fig:theoplots}
  \endminipage\hfill
\end{figure}

\begin{figure} [!htb] 
  \minipage{0.9\textwidth}
  \includegraphics[width=\linewidth]{}
  \caption{Octave voltage phase}
  \label{fig:theoplots}
  \endminipage\hfill
\end{figure}

In the following tables, the input and ouput inpedances  calculated at the central frequency from both analysis are presented.
\FloatBarrier
\begin{table}[h]
  \centering
  \begin{tabular}{|c|c|}
    \hline    
    \input{ }
    \hline
  \end{tabular}
  \caption{Input Impedance by Ngspice}
  \label{tab:Spice1}
\end{table}
\FloatBarrier   

\FloatBarrier
\begin{table}[h]
  \centering
  \begin{tabular}{|c|c|c|c|}
    \hline    
    \input{}
    \hline
  \end{tabular}
  \caption{Input and Output Impedances by Octave}
  \label{tab:Spice1}
\end{table}
\FloatBarrier   

The cost of the design is: \input{}\\


