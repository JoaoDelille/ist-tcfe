\section{Comparisons}
\label{sec:comparsisons}

In this section, a comparision between the ngspice and Octave results is made. Moreover, we delve into the reasons regarding the diffences and similarities observed.\\
Starting with the frequency response, the gain and phase are presented bellow.\\
\FloatBarrier
Gain:
\begin{figure} [htb] 
	\begin{subfigure}[b]{0.5\textwidth}
		\centering
  		\includegraphics[trim={0 45px 0 230px},width=\linewidth]{resources/gain.pdf}
  		\caption{NGspice voltage gain}
	\end{subfigure}
  	\begin{subfigure}[b]{0.5\textwidth}
  		\centering
 		 \includegraphics[width=\textwidth]{resources/T(s).png}
 		 \caption{Octave voltage gain}
  	\end{subfigure}
\end{figure}
\FloatBarrier
Phase:
\begin{figure} [htb] 
	\begin{subfigure}[b]{0.5\textwidth}
 		 \includegraphics[trim={0 45px 0 230px},width=\linewidth]{resources/phase.pdf}
  		\caption{NGspice voltage phase}
 		\label{fig:theoplots}
	\end{subfigure}
  	\begin{subfigure}[b]{0.5\textwidth}
  		\includegraphics[width=\textwidth]{resources/phase_deg.png}
 		 \caption{Octave voltage phase}
 		 \label{fig:theoplots}
  	\end{subfigure}
\end{figure}
\FloatBarrier
It is possible to observe that the maximum gain is achieved at around 1000Hz, as it was intended, both in the simulation and in the Octave model.\\
Regarding the phase, one can note that the phase of the simulated circuit has a less sharp slope than the Octave one. It is also noticeable that the ngspice simulation shows smaller frequency bands in which the phase has a small slope, whereas Octave showed significant plateaus.\\
The differences in the gain and phase can be linked to the use of a more complex OP-AMP model in ngspice whereas an ideal model was modelled in Octave. The simulated one does not have infinite input and zero output impedances. Furthermore, the amplification, Av, is not infinite. The theoretical model also ignores any capacitive and inductive effects as well as non-linearities in the OP-AMP, which leads the frequency response to be seemingly inconsistent in contrast to the idealized smooth gain and phase curves of the theoretical model. Taking these distinctions into consideration, it is expected to have diferences between the models.\\
In the following tables, the input and ouput impedances calculated at the central frequency from both analyses are presented.
\begin{table}[h!]
  \centering
  \begin{tabular}{|l|l|l|}
    \hline    
    \input{resources/inputimped_tab.tex}
    \input{resources/outputimped_tab.tex}
    \input{resources/Z.tex}
    \hline
  \end{tabular}
  \caption{Impedances}
\end{table}
\FloatBarrier   
The impedances are within the same order of magniture in each analysis. However, the output impedance is higher in the theoretical analysis and, in contrast, the imput impedance is higher in the simulated one. Such disparity can be linked to the diferent OP-AMP models used, since the theoretical analysis one is perfect. This implies that the true output and input impedances of the OP AMP weren't taken into account, since the calculations used an infinite imput impedance and a null output one. The real OP AMP has a significantly lower input impedance, since it's finite, and a higher output impedance. This fact explains the disparity. 
%The cost of the design is: \input{}\\


