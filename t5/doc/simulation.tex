\section{Simulation Analysis}
\label{sec:simulation} 

From operating point analysis, we got the following results:
%\FloatBarrier
%\begin{table}[h]
%  \centering
%  \begin{tabular}{|c|c|c|c|c|}
%    \hline    
%    \input{resources/frequencies_tab.tex}

%  \end{tabular}
%  \caption{frequencies}
%  \label{tab:Spice1}
%\end{table}
%\FloatBarrier 


%\begin{table}[h]
%  \centering
%  \begin{tabular}{|c|c|c|c|c|}
%    \hline    
%    \input{resources/maxgain_tab.tex}

%  \end{tabular}
%  \caption{gain}
%  \label{tab:Spice1}
%\end{table}


\FloatBarrier  
\begin{table}[!htb]

\parbox{.45\linewidth}{
\centering
\begin{tabular}{|l|l|}
    \hline
    \input{resources/frequencies_tab.tex}

  \end{tabular}
  \caption{frequencies}
}

\hfill
\parbox{.45\linewidth}{
\centering
\begin{tabular}{|l|l|}
    \hline    
    \input{resources/maxgain_tab.tex}

  \end{tabular}
  \caption{gain}
}

\end{table}
\FloatBarrier  


\FloatBarrier  
\begin{table}[!htb]
\parbox{.45\linewidth}{
\centering
\begin{tabular}{|l|l|}
    \hline    
    \input{resources/deviations_tab.tex}
  \end{tabular}
  \caption{deviations}
}

\hfill
\parbox{.45\linewidth}{
\centering
\begin{tabular}{|l|l|}
    \hline    
    \input{resources/merit_tab.tex}
  \end{tabular}
  \caption{merit}
  \label{tab:Spice1}
}
\end{table}
\FloatBarrier  



%\FloatBarrier  
%\begin{table}[h]
%  \centering
%  \begin{tabular}{|c|c|c|c|c|}
%    \hline    
%    \input{resources/deviations_tab.tex}

%  \end{tabular}
%  \caption{deviations}
%  \label{tab:Spice1}
%\end{table}
%\FloatBarrier 

%\FloatBarrier  
%\begin{table}[h]
%  \centering
%  \begin{tabular}{|c|c|c|c|}
%    \hline    
%    \input{resources/merit_tab.tex}

%  \end{tabular}
%  \caption{merit}
  %\label{tab:Spice1}
%\end{table}
%\FloatBarrier  

The input and output impedances for the central frequency obtained can be observed in the following tables.

%\FloatBarrier
%\begin{table}[h]
 % \centering
 % \begin{tabular}{|c|c|c|}
 %   \hline    
%    \input{resources/inputimped_tab.tex}
%  \end{tabular}
%  \caption{Input Impedace}
%  \label{tab:Spice1}
%\end{table}
%\FloatBarrier 

%\FloatBarrier  
%\begin{table}[h]
%  \centering
%  \begin{tabular}{|c|c|}
%    \hline    
%    \input{resources/outputimped_tab.tex}

%  \end{tabular}
 % \caption{Output Impedace}
%  \label{tab:Spice1}
%\end{table}
%\FloatBarrier  









\begin{table}[!htb]


\parbox{.45\linewidth}{
  \centering
  \begin{tabular}{|l|l|}
    \hline    
    \input{resources/outputimped_tab.tex}
  \end{tabular}
  \caption{Output Impedace}
}

\hfill
\parbox{.45\linewidth}{
  \centering
  \begin{tabular}{|l|l|}
    \hline    
    \input{resources/inputimped_tab.tex}
  \end{tabular}
  \caption{Input Impedace}
}

\end{table}














\FloatBarrier  
\begin{figure} [!htb] 
  \includegraphics[trim={0 45px 0 230px},width=\linewidth]{resources/gain.pdf}
  \caption{Gain}
  \label{fig:theoplots}
  \hfill
\end{figure}
\FloatBarrier  

\FloatBarrier
\begin{figure} [!htb] 
  \minipage{0.9\textwidth}
  \includegraphics[width=\linewidth]{resources/phase.pdf}
  \caption{Phase}
  \label{fig:theoplots}
  \endminipage
  \hfill
\end{figure}
\FloatBarrier


%Escrever como as incrementações nos valores influenciam por exemplo a utilização de um condensador e consequentemente a resistência do mesmo no gain voltage
