\section{Simulation Analysis}
\label{sec:simulation} 

From operating point analysis, we got the following results:
\FloatBarrier
\begin{table}[h]
  \centering
  \begin{tabular}{|c|c|c|c|c|}
    \hline    
    \input{merit_tab.tex}
    \hline
  \end{tabular}
  \caption{merit}
  \label{tab:Spice1}
\end{table}
\FloatBarrier  

The input and output impedances for the central frequency obtained can be observed in the following tables.

\FloatBarrier
\begin{table}[h]
  \centering
  \begin{tabular}{|c|c|}
    \hline    
    \input{inputimp_tab.tex}
    \hline
  \end{tabular}
  \caption{Input Impedace}
  \label{tab:Spice1}
\end{table}
\FloatBarrier 
  
\FloatBarrier
\begin{table}[h]
  \centering
  \begin{tabular}{|c|c|}
    \hline    
    \input{outputimp_tab.tex}
    \hline
  \end{tabular}
  \caption{Input Impedace}
  \label{tab:Spice1}
\end{table}
\FloatBarrier  

The frequency response analysis yielded the following results:
\begin{figure} [!htb] 
  \minipage{0.9\textwidth}
  \includegraphics[width=\linewidth]{gain.pdf}
  \caption{Gain}
  \label{fig:theoplots}
  \endminipage\hfill
\end{figure}

\begin{figure} [!htb] 
  \minipage{0.9\textwidth}
  \includegraphics[width=\linewidth]{gain.pdf}
  \caption{Phase}
  \label{fig:theoplots}
  \endminipage\hfill
\end{figure}



%Escrever como as incrementações nos valores influenciam por exemplo a utilização de um condensador e consequentemente a resistência do mesmo no gain voltage
