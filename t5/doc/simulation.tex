\section{Simulation Analysis}
\label{sec:simulation} 



From operating point analysis we got the following results:
\FloatBarrier
\begin{table}[h]
  \centering
  \begin{tabular}{|c|c|c|c|c|}
    \hline    
    \input{merit_tab.tex}
    \hline
  \end{tabular}
  \caption{merit}
  \label{tab:Spice1}
\end{table}
\FloatBarrier  

And now for the impedances values, the input and output impedances, respectively:

\FloatBarrier
\begin{table}[h]
  \centering
  \begin{tabular}{|c|c|}
    \hline    
    \input{inputimp_tab.tex}
    \hline
  \end{tabular}
  \caption{Input Impedace}
  \label{tab:Spice1}
\end{table}
\FloatBarrier 
  
\FloatBarrier
\begin{table}[h]
  \centering
  \begin{tabular}{|c|c|}
    \hline    
    \input{outputimp_tab.tex}
    \hline
  \end{tabular}
  \caption{Input Impedace}
  \label{tab:Spice1}
\end{table}
\FloatBarrier  


\begin{figure} [!htb] 
  \minipage{0.9\textwidth}
  \includegraphics[width=\linewidth]{gain.pdf}
  \caption{Incremental Gain}
  \label{fig:theoplots}
  \endminipage\hfill
\end{figure}



%Escrever como as incrementações nos valores influenciam por exemplo a utilização de um condensador e consequentemente a resistência do mesmo no gain voltage
