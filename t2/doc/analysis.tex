\section{Theoretical Analysis}
\label{sec:analysis}

\par As stated earlier, in this section, we go through various steps to obtained the total solution of the circuit and present the theoretical frequency response analysis.



\subsection{Node analysis for $t<0$} 

\par The first point of the Theoretical analysis requested for a study of the node voltages for $t<0$ which affect the type of signal that pass through the capacitor. 

\begin{align*}  
v_s(t)=V_s*u(-t)+sin(2*pi*f*t)*u(t)\\   
\end{align*} 

where the u(t) function controls the type of signal that the source provides, which means when $t<0$, a constant signal is made, when the $t>0$ or equal to 0, a sinusoidal signal is propagated.  
In this case, the voltage is constant and consequently the capacitor behaves like an open circuit. Then a study of the nodes voltages was made, applying the node method with the KCL equations and transforming to a matrix form, we get:

$$
\begin{bmatrix} 
   1     & 0               & 0    & 0           & 0    & 0          & 0    \\
   1/R1  & -1/R1-1/R2-1/R3 & 1/R2 & 1/R3        & 0    & 0          & 0    \\
   0     & 0               & 0    & 0           & 0    & -1/R6-1/R7 & 1/R7 \\
   0     & 0               & 0    & 1           & 0    & Kd/R6      & -1   \\
   0     & Kb              & 0    & -1/R5-Kb    & 1/R5 & 0          & 0    \\
   0     & -1/R2-Kb        & 1/R2 & Kb          & 0    & 0          & 0    \\
   -1/R1 & 1/R1            & 0    & 1/R4        & 0    & 1/R6       & 0    \\
\end {bmatrix} 
\begin{bmatrix}
V1 \\ V2 \\ V3 \\ V5 \\ V6 \\ V7 \\ V8
\end {bmatrix} 
=
\begin{bmatrix} 
Vs \\ 0 \\ 0 \\ 0 \\ Id \\ 0 \\ 0
\end {bmatrix} 
$$ 


\vspace{0.5cm}
And finally the results obtained, using Octave to solve the matrix, are in the table down below. 


\FloatBarrier
\begin{table}[h]
  \centering
  \begin{tabular}{|c|c|c|c|c|c|c|}
    \hline    
    \input{nodeVolt}
    \hline
  \end{tabular}
  \caption{Nodal method}
  \label{tab:nodal}
\end{table}
\FloatBarrier 

And for the currents in all branches we apllied directly for each component the Ohm's Law. 

\FloatBarrier
\begin{table}[h]
  \centering
  \begin{tabular}{|c|c|c|c|c|c|c|}
    \hline    
    \input{nodeCurr}
    \hline
  \end{tabular}
  \caption{Nodal method}
  \label{tab:nodal}
\end{table}
\FloatBarrier 


\subsection{Point (2)} 

The second point of this analysis consists on computing the equivalent resistance seen through the capacitor. We made the node analysis, making $V_s=0$ and replacing the capacitor with a voltage source named $V_6-V_8$, calculated in the previous step.
These procedure is necessary to make sure that the voltage in the capacitor is continuos.   

The matriz used for the node analysis:
$$
\begin{bmatrix} 
   1/R1           & 0       & 1/R4    & 0    & 1/R6         & 0       \\
  -1/R2-Kb        & 1/R2    & Kb      & 0    & 0            & 0       \\
   1/R1+1/R2+1/R3 & -1/R2   & -1/R3   & 0    & 0            & 0       \\
   0              & 0       & 0       & 0    & 1/R6 + 1/R7  & -1/R7   \\
   0              & 0       & 0       & 1    & 0            & -1      \\
   0              & 0       & -1      & 0    & Kd/R6        & 0       \\
\end {bmatrix} 
\begin{bmatrix}
V2 \\ V3 \\ V5 \\ V6 \\ V7 \\ V8
\end {bmatrix} 
=
\begin{bmatrix} 
0 \\ 0 \\ 0 \\ 0 \\ V6-V8 \\ 0
\end {bmatrix} 
$$ 

Where: 
\begin{align*} 
&I_x=\frac{(V_6-V_5)}{R_5} - K_b*(V2-V5)\\
\\    
&R_{eq}=\frac{(V_8-V_6)}{I_X}\\
\\
&timecte=R_{eq}*C\\
\end{align*}


The results for these procedure are presented down bellow: 

\vspace{0.2cm}
\FloatBarrier
\begin{table}[h]
  \centering
  \begin{tabular}{|c|c|c|c|c|c|c|}
	\hline
    \input{condensador}
	\hline
  \end{tabular}
  \caption{Nodal method for point (2)}
  \label{tab:nodal}
\end{table}
\FloatBarrier 
  

\subsection{Natural solution} 

We used the Thevenin circuit calculated in the previous point to calculate the condenser discharge. The equation for the natural regime in RC circuits is: 

\begin{equation}
  V_{6n}=V_6*e^{-\frac{t}{RC}}
  \label{eq:kvl}
\end{equation}   

The initial voltage in node 6 was obtained in the previous point when we compute the equivalent resistor along with charecteristc time. 

From this equation we computed the natural response in node 6, in [0,0.02] seconds.  
\FloatBarrier
\begin{figure}
  \includegraphics[width=\linewidth]{natural.png}
  \caption{Natural solution of V6, Theo-3}
  \label{fig:natural}
\end{figure}
\FloatBarrier

\subsection{Forced solution} 

In this point the forced current was represented with a phasor, which is a complex number.Then the capacitor was replaced by its impedance, $\frac{1}{j\omega C}$.The nodes were analyzed in the same way as in pont 1.The time function will then be matched to the actual part of the phasor multiplied by $e^{\omega*t}$. 


\vspace{0.2cm}
\FloatBarrier
\begin{table}[h]
  \centering
  \begin{tabular}{|c|c|}
   \hline
    \input{phasor}
	\hline
  \end{tabular}
  \caption{Nodal method for point (2)}
  \label{tab:nodal}
\end{table}
\FloatBarrier 





%\FloatBarrier
%\begin{table}[h]
%  \centering
%  \begin{tabular}{|c|c|c|c|c|c|c|}
%    \hline    
%    
 V(1) & V(2) & V(3) & V(4) & V(5) & V(6) & V(7) \\ 
 5.24841 V   & 4.9997 V  & 4.47559 V  & 5.03486 V  & 5.80216 V  & -1.97226 V  & -2.96144 V\\

%    \hline
%  \end{tabular}
%  \caption{Nodal method}
%  \label{tab:nodal}
%\end{table}
%\FloatBarrier

%\FloatBarrier
%\begin{table}[h]
%  \centering
%  \begin{tabular}{|c|c|c|c|}
%   \hline    
%   \input{mesh}
%    \hline
%  \end{tabular}
%  \caption{Mesh method}
%  \label{tab:mesh}
%\end{table}
%\FloatBarrier
