\section{Theoretical Analysis}
\label{sec:analysis}

%As stated earlier, in this section, the theoretical analysis of the laboratory is made. These results were all obtained through GNU Octave.



\begin{Capacitor, C, as a open circuit} 
%The first point of the Theoretical analysis requested "asked" for a study of the node voltages for t<0, that is, for a constant voltage signal, at the voltage source, Vs. 


\end{align*}


[filler]
[filler]


$$
\begin{bmatrix} 
[filler]
[filler]
\end {bmatrix} 
\begin{bmatrix} 
[filler]
[filler]
\end {bmatrix} 
=
\begin{bmatrix} 
[filler]
[filler]
\end {bmatrix} 
$$


[filler]
[filler]

\begin{align*} 
[filler]
[filler]
\end{align*}

 

[filler]
[filler]



\begin{align*} 
[filler]
[filler]
\end{align*}

[filler]
[filler]

\begin{align*} 
[filler]
[filler]
\end{align*}

[filler]
[filler]


$$
\begin{bmatrix} 
[filler]
[filler]
\end {bmatrix} 
\begin{bmatrix}
[filler]
[filler]
\end{bmatrix}
=
\begin{bmatrix}
[filler]
[filler]
\end{bmatrix}
$$


[filler]
[filler]
[filler]
[filler]


%\FloatBarrier
%\begin{table}[h]
%  \centering
%  \begin{tabular}{|c|c|c|c|c|c|c|}
%    \hline    
%    
 V(1) & V(2) & V(3) & V(4) & V(5) & V(6) & V(7) \\ 
 5.24841 V   & 4.9997 V  & 4.47559 V  & 5.03486 V  & 5.80216 V  & -1.97226 V  & -2.96144 V\\

%    \hline
%  \end{tabular}
%  \caption{Nodal method}
%  \label{tab:nodal}
%\end{table}
%\FloatBarrier

%\FloatBarrier
%\begin{table}[h]
%  \centering
%  \begin{tabular}{|c|c|c|c|}
%   \hline    
%   \input{mesh}
%    \hline
%  \end{tabular}
%  \caption{Mesh method}
%  \label{tab:mesh}
%\end{table}
%\FloatBarrier
