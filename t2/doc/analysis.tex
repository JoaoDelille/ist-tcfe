\section{Theoretical Analysis}
\label{sec:analysis}

\par As stated earlier, in this section, the theoretical analysis of the laboratory is made. These results were all obtained through GNU Octave.



\subsection{Point (1)} 

\par The first point of the Theoretical analysis requested for a study of the node voltages for t<0 which affect the type of signal that pass . 

\begin{align*}  
v_s(t)=V_s*u(-t)+sin(2*pi*f*t)*u(t)\\   
\end{align*} 

where the u(t) function controls the type of signal that the source provides, which means when $t<0$, a constant signal is made, when the $t>0$ or equal to 0, a sinusoidal signal is propagated.  
In this case, the voltage is constant and consequently the capacitor behaves like an open circuit. Then a study of the nodes voltages was made, applying the node method with the following KCL equations: 

\begin{align*} 
&(V1-V2)/R1+(V3-V2)/R2 = (V2-V5)/R3 \\ 
&-V7/R6 = (V7-V8)/R7\\
&(V3-V2)/R2 = Kb(V2-V5) \\
&(V6-V5)/R5+Kb(V2-V5) = Id \\
\end{align*} 

And three additional equations in order to get a total of seven equations: 

\begin{align*} 
&V1=Vs \\ 
&V5-V8=-Kd/R6\\ 
&(V2-V1)/R1+V5/R4+V7/R6=0
\end{align*} 

This was transformed in matrix form:

$$
\begin{bmatrix} 
   1     & 0               & 0    & 0           & 0    & 0          & 0    \\
   1/R1  & -1/R1-1/R2-1/R3 & 1/R2 & 1/R3        & 0    & 0          & 0    \\
   0     & 0               & 0    & 0           & 0    & -1/R6-1/R7 & 1/R7 \\
   0     & 0               & 0    & 1           & 0    & Kd/R6      & -1   \\
   0     & Kb              & 0    & -1/R5-Kb    & 1/R5 & 0          & 0    \\
   0     & -1/R2-Kb        & 1/R2 & Kb          & 0    & 0          & 0    \\
   -1/R1 & 1/R1            & 0    & 1/R4        & 0    & 1/R6       & 0    \\
\end {bmatrix} 
\begin{bmatrix}
V1 \\ V2 \\ V3 \\ V5 \\ V6 \\ V7 \\ V8
\end {bmatrix} 
=
\begin{bmatrix} 
Vs \\ 0 \\ 0 \\ 0 \\ Id \\ 0 \\ 0
\end {bmatrix} 
$$ 

%\vspace{0.5cm}
%And finally the results obtained, using Octave to solve the matrix, are in the table down below. 

%\FloatBarrier
%\begin{table}[h]
%  \centering
%  \begin{tabular}{|c|c|c|c|c|c|c|}
%    \hline    
%    
 V(1) & V(2) & V(3) & V(4) & V(5) & V(6) & V(7) \\ 
 5.24841 V   & 4.9997 V  & 4.47559 V  & 5.03486 V  & 5.80216 V  & -1.97226 V  & -2.96144 V\\

%    \hline
%  \end{tabular}
%  \caption{Nodal method}
%  \label{tab:nodal}
%\end{table}
%\FloatBarrier

\subsection{Point (2)} 

The second point of this analysis consists on computing the equivalent resistance, the current that flows through the capacitor and the time constant, using v6 and v8 from the previous point. 
These procedure is necessary since it allows us to study when the capacitor is under sinusoidal signal conditions.   
The node analysis was applied, using the following equations:
 
\begin{align*} 
&V2/R1+V5/R4+V7/R6=0//
&(V3-V2)/R2+(V5-V2)Kb//
&V2/R1+(V2-V3)/R2+(V2-V5)/R3=0//
&V7/R6+(V7-V8)/R7// 
&V6-V8=8.763595// 
&V5+(V7*Kd)/R6-V8=0//
\end{align*}

In the matrix from, we obtained:  

$$
\begin{bmatrix} 
   1/R1           & 0       & 1/R4    & 0    & 1/R6         & 0       \\
  -1/R2-Kb        & 1/R2    & Kb      & 0    & 0            & 0       \\
   1/R1+1/R2+1/R3 & -1/R2   & -1/R3   & 0    & 0            & 0       \\
   0              & 0       & 0       & 0    & 1/R6 + 1/R7  & -1/R7   \\
   0              & 0       & 0       & 1    & 0            & -1      \\
   0              & 0       & -1      & 0    & Kd/R6        & 0       \\
\end {bmatrix} 
\begin{bmatrix}
V2 \\ V3 \\ V5 \\ V6 \\ V7 \\ V8
\end {bmatrix} 
=
\begin{bmatrix} 
0 \\ 0 \\ 0 \\ 0 \\ 8.763595 \\ 0
\end {bmatrix} 
$$ 

The results for these procedure are presented down bellow: 

%%colocar tabela de node_alinea.tex aqui.  

\subsection{Point (3)} 

We used the theremin circuit calculated in the previous point to calculate the condenser discharge, whose function can be deducted from the meshes law and solves the resulting differential equation. This equation is equal to:  

\begin{equation}
  V_{6n}= V6e^{-\frac{t}{RC}}
  \label{eq:kvl}
\end{equation}  

%dar plot correspondido 

\subsection{Point (4)} 

In this point the forced current was represented with a phasor, which is a complex number.Then the capacitor was replaced by its impedance.The nodes were analyzed in the same way as in pont 1.The time function will then be matched to the actual part of the phasor multiplied by $e^{\omega*t}$. 

\begin{align*} 
[filler]
[filler]
\end{align*}

[filler]
[filler]

\begin{align*} 
[filler]
[filler]
\end{align*}

[filler]
[filler]


$$
\begin{bmatrix} 
[filler]
[filler]
\end {bmatrix} 
\begin{bmatrix}
[filler]
[filler]
\end{bmatrix}
=
\begin{bmatrix}
[filler]
[filler]
\end{bmatrix}
$$


[filler]
[filler]
[filler]
[filler]


%\FloatBarrier
%\begin{table}[h]
%  \centering
%  \begin{tabular}{|c|c|c|c|c|c|c|}
%    \hline    
%    
 V(1) & V(2) & V(3) & V(4) & V(5) & V(6) & V(7) \\ 
 5.24841 V   & 4.9997 V  & 4.47559 V  & 5.03486 V  & 5.80216 V  & -1.97226 V  & -2.96144 V\\

%    \hline
%  \end{tabular}
%  \caption{Nodal method}
%  \label{tab:nodal}
%\end{table}
%\FloatBarrier

%\FloatBarrier
%\begin{table}[h]
%  \centering
%  \begin{tabular}{|c|c|c|c|}
%   \hline    
%   \input{mesh}
%    \hline
%  \end{tabular}
%  \caption{Mesh method}
%  \label{tab:mesh}
%\end{table}
%\FloatBarrier
