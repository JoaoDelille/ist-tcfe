\section{Theoretical Analysis}
\label{sec:analysis}

\par As stated earlier, in this section, the theoretical analysis of the laboratory is made. These results were all obtained through GNU Octave.



\subsection{Capacitor as an open circuit} 

\par The first point of the Theoretical analysis requested for a study of the node voltages for t<0 which affect the type of signal that pass . 

\begin{align*}  
v_s(t)=V_s*u(-t)+sin(2*pi*f*t)*u(t)\\   
\end{align*} 

where the u(t) function controls the type of signal that the source provides, which means when $t<0$, a constant signal is made, when the $t>0$ or equal to 0, a sinusoidal signal is propagated.  
In this case, the voltage is constant and consequently the capacitor behaves like an open circuit. Then a study of the nodes voltages was made, applying the node method with the following KCL equations: 

\begin{align*} 
&(V1-V2)/R1+(V3-V2)/R2 = (V2-V5)/R3 \\ 
&-V7/R6 = (V7-V8)/R7\\
&(V3-V2)/R2 = Kb(V2-V5) \\
&(V6-V5)/R5+Kb(V2-V5) = Id \\
\end{align*} 

And three additional equations in order to get a total of seven equations: 

\begin{align*} 
&V1=Vs \\ 
&V5-V8=-Kd/R6\\ 
&(V2-V1)/R1+V5/R4+V7/R6=0
\end{align*} 

This was transformed in matrix form:

$$
\begin{bmatrix} 
   1     & 0               & 0    & 0           & 0    & 0          & 0    \\
   1/R1  & -1/R1-1/R2-1/R3 & 1/R2 & 1/R3        & 0    & 0          & 0    \\
   0     & 0               & 0    & 0           & 0    & -1/R6-1/R7 & 1/R7 \\
   0     & 0               & 0    & 1           & 0    & Kd/R6      & -1   \\
   0     & Kb              & 0    & -1/R5-Kb    & 1/R5 & 0          & 0    \\
   0     & -1/R2-Kb        & 1/R2 & Kb          & 0    & 0          & 0    \\
   -1/R1 & 1/R1            & 0    & 1/R4        & 0    & 1/R6       & 0    \\
\end {bmatrix} 
\begin{bmatrix}
V1 \\ V2 \\ V3 \\ V5 \\ V6 \\ V7 \\ V8
\end {bmatrix} 
=
\begin{bmatrix} 
Vs \\ 0 \\ 0 \\ 0 \\ Id \\ 0 \\ 0
\end {bmatrix} 
$$ 

\vspace{0.5cm}
And finally the results obtained, using Octave to solve the matrix, are in the table down below. 

\FloatBarrier
\begin{table}[h]
  \centering
  \begin{tabular}{|c|c|c|c|c|c|c|}
    \hline    
    
 V(1) & V(2) & V(3) & V(4) & V(5) & V(6) & V(7) \\ 
 5.24841 V   & 4.9997 V  & 4.47559 V  & 5.03486 V  & 5.80216 V  & -1.97226 V  & -2.96144 V\\

    \hline
  \end{tabular}
  \caption{Nodal method}
  \label{tab:nodal}
\end{table}
\FloatBarrier


[filler]
[filler]

\begin{align*} 
[filler]
[filler]
\end{align*}

 

[filler]
[filler]



\begin{align*} 
[filler]
[filler]
\end{align*}

[filler]
[filler]

\begin{align*} 
[filler]
[filler]
\end{align*}

[filler]
[filler]


$$
\begin{bmatrix} 
[filler]
[filler]
\end {bmatrix} 
\begin{bmatrix}
[filler]
[filler]
\end{bmatrix}
=
\begin{bmatrix}
[filler]
[filler]
\end{bmatrix}
$$


[filler]
[filler]
[filler]
[filler]


%\FloatBarrier
%\begin{table}[h]
%  \centering
%  \begin{tabular}{|c|c|c|c|c|c|c|}
%    \hline    
%    
 V(1) & V(2) & V(3) & V(4) & V(5) & V(6) & V(7) \\ 
 5.24841 V   & 4.9997 V  & 4.47559 V  & 5.03486 V  & 5.80216 V  & -1.97226 V  & -2.96144 V\\

%    \hline
%  \end{tabular}
%  \caption{Nodal method}
%  \label{tab:nodal}
%\end{table}
%\FloatBarrier

%\FloatBarrier
%\begin{table}[h]
%  \centering
%  \begin{tabular}{|c|c|c|c|}
%   \hline    
%   \input{mesh}
%    \hline
%  \end{tabular}
%  \caption{Mesh method}
%  \label{tab:mesh}
%\end{table}
%\FloatBarrier
