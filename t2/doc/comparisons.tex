\section{Comparisons}
\label{sec:comparisons}

In this section the objective is to compare the same from GNU Octave and Ngspice. 
Starting with the values of the node voltages and all the branches currennts we have: 

\vspace{0.5cm}
For point (1):
Octave: 
\FloatBarrier
\begin{table}[h]
  \centering
  \begin{tabular}{|c|c|c|c|c|c|c|}
    \hline    
    \input{nodeVolt}
    \hline
  \end{tabular}
  \caption{Nodal method}
  \label{tab:nodal}
\end{table}
\FloatBarrier  

\FloatBarrier
\begin{table}[h]
  \centering
  \begin{tabular}{|c|c|c|c|c|c|c|}
    \hline    
    \input{nodeCurr}
    \hline
  \end{tabular}
  \caption{Nodal method}
  \label{tab:nodal}
\end{table}
\FloatBarrier 

Ngspice: 



%ver o que se passa aqui valores completamnete diferentes!!! 

\vspace{0.5cm} 
For point (2): 
Octave:
\vspace{0.2cm}
\FloatBarrier
\begin{table}[h]
  \centering
  \begin{tabular}{|c|c|c|c|c|c|c|}
    \hline    
    \input{condensador}
    \hline
  \end{tabular}
  \caption{Nodal method for point (2)}
  \label{tab:nodal}
\end{table}
\FloatBarrier   

Ngspice:



% grafico ngspice the-3 falta do octave para comparar





%%% provavelmente vai para theoretical analysis
\vspace{0.5cm} 
For point (2): 
Octave:
\vspace{0.2cm}
\FloatBarrier
\begin{table}[h]
  \centering
  \begin{tabular}{|c|c|c|c|c|c|c|}
    \hline    
    \input{phasor}
    \hline
  \end{tabular}
  \caption{Theo-4}
  \label{tab:nodal}
\end{table}
\FloatBarrier 



















