\section{Operating Point Analysis}
\label{sec:simulation}

To realize the computational simulation of the circuit analysis, ngspice was used. A script containing the netlist used by the program was created to perform a DC (Operating Point) analysis, and print the results to the tables down below.\\ 
It is notable that an extra node, node $7.1$ is present in the tables, as well as a short circuited voltage source, $V_{e}$ connecting node $7.1$ and $7.2$. This procedure doesn't affect the circuit electrically, but is required for the correct funtioning of ngspice. For the correct simulation of the current dependent voltage source $V_{c}$, the current it depends on must be read from a shorted voltage source, as ngspice cannot read it from a resistor. Since this shorted voltage source is in series with the resistor $R_{6}$, the current is the same in both elements, and thus can be read from either of them.\\ 

\subsection{Operating point analysis} 

The tables below show the simulated operating point results for $t<0$ and $t=0$, respectively. 
\FloatBarrier
\begin{table}[h]
  \centering
  \begin{tabular}{|l|r|}
    \hline    
    {\bf Name} & {\bf Value [A or V]} \\ \hline
    \input{2a_tab}
  \end{tabular}
  \caption{Simulated values for $t<0$ from Ngspice}
  \label{tab:op}
\end{table}
\FloatBarrier

\FloatBarrier
\begin{table}[h]
  \centering
  \begin{tabular}{|l|r|}
    \hline    
    {\bf Name} & {\bf Value [A or V]} \\ \hline
    \input{2b_tab}
  \end{tabular}
  \caption{Simulated values for $t=0$ from Ngspice}
  \label{tab:op}
\end{table}
\FloatBarrier


\subsection{Natural response}
In this subsection we simulate the natural response of the circuit in the time interval [0,20]ms, using essentially the transient analysis simulation in NGspice. 

\FloatBarrier
\begin{figure}[h] \centering
\includegraphics[width=0.6\linewidth]{plotnatural.pdf}
\caption{Natural solution of V6, Sim-3}
\label{fig:plotnatural}
\end{figure}
\FloatBarrier

\subsection{Natural and forced response} 

In this subsectio we repeat the previous step but now with the sinusoidal voltage source $V_(t)$, for a frequency $f=1kHz$. 

\FloatBarrier
\begin{figure}[h] \centering
\includegraphics[width=0.6\linewidth]{plotfinal.pdf}
\caption{Simulated response for natural and forced solution for $V-6$, Sim-4}
\label{fig:plotfinal}
\end{figure}
\FloatBarrier 

\subsection{Frequency response} 

We simulated the frequency response to $V_s$, $V_6$ and $V_C$, Voltage in the capacitor. We plotted the magnitude, in dB, and phase, in degrees, of the three voltages in relation to a variation of frequency. The frequency varies between 0.1Hz to 1MHz 

\FloatBarrier
\begin{figure}[h] \centering
\includegraphics[width=0.6\linewidth]{plotmagnitude.pdf}
\caption{Magnitude of $V_s$, $V_c$ and $V_6$, Sim-5 }
\label{fig:plotmagnitude}
\end{figure}
\FloatBarrier  

\FloatBarrier
\begin{figure}[h] \centering
\includegraphics[width=0.6\linewidth]{plotfase.pdf}
\caption{Phase of $V_s$, $V_c$ and $V_6$, Sim-5}
\label{fig:plotfase}
\end{figure}
\FloatBarrier 


%\FloatBarrier
%\begin{table}[h]
%  \centering
%  \begin{tabular}{|l|r|}
%    \hline    
%    {\bf Name} & {\bf Value [A or V]} \\ \hline
%    low & 2.513670e+00\\ \hline
cost & 7.109638e+03\\ \hline
gain & 1.487926e+01\\ \hline
bandwidth & 3.136186e+06\\ \hline
merit & 2.611124e+03\\ \hline

%  \end{tabular}
%  \caption{Simulated values from Ngspice}
%  \label{tab:op}
%\end{table}
%\FloatBarrier


