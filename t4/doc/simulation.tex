\section{Simulation Analysis}
\label{sec:simulation} 

In the simulation study of an amplifier circuit, the Ngspice software was used, where some modifications were made to the script initially made available in order to obtain the best merit in this work. For this and also as a requirement, the NPN and PNP models were used for the transistors, referring to the Gain stage and the Output stage, respectively.

From operating point analysis we got the following results:
\FloatBarrier
\begin{table}[h]
  \centering
  \begin{tabular}{|c|c|c|c|c|}
    \hline    
    low & 2.513670e+00\\ \hline
cost & 7.109638e+03\\ \hline
gain & 1.487926e+01\\ \hline
bandwidth & 3.136186e+06\\ \hline
merit & 2.611124e+03\\ \hline

    \hline
  \end{tabular}
  \caption{Operating Point Analysis}
  \label{tab:Spice1}
\end{table}
\FloatBarrier  

And now for the impedances values, the input and output impedances, respectively:

\FloatBarrier
\begin{table}[h]
  \centering
  \begin{tabular}{|c|}
    \hline    
    zin & 2.738309e+00\\ \hline

    \hline
  \end{tabular}
  \caption{Input Impedace}
  \label{tab:Spice1}
\end{table}
\FloatBarrier   

%\FloatBarrier
%\begin{table}[h]
%  \centering
%  \begin{tabular}{|c|}
%    \hline    
%    \input{output_tab.tex}
%    \hline
%  \end{tabular}
%  \caption{Voltages}
%  \label{tab:Spice1}
%\end{table}
%\FloatBarrier  




%Escrever como as incrementações nos valores influenciam por exemplo a utilização de um condensador e consequentemente a resistência do mesmo no gain voltage
