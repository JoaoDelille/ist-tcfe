\section{Conclusion}
In this laboratory we were able to apply the Kirchhoff's Laws and Ohm's Law in two structured ways to analyse a circuit with independent and dependent voltage and current sources, as well as linear resistors. In these methods, we obtained a system of linear equations that we could solve using Octave to aquire either the mesh currents, or the relative voltage at any node.\\
We also used a simulation software in order to analyse it in a diferent manner: through operating point analysis. This way, we were able to create a model that substitutes real life circuits but that still let us compare the results of our theoretical analysis.\\
Using ngspice it took us very little time to write down and simulate the circuit, obtaining results almost instantly once the netlist was finalized. On the other hand, applying KKirchhoff's and Ohm's laws by hand to determine the necessary linear equations proved to be time consuming and error prone.\\
Given the simulator's speed, it became clear that it would be the preferred method for analyzing any sizeable circuit. Even though it can introduce a small calculation error even in simple circuits like this, that error is small enough that it can be ignored for most circuit designs and applications in practice.
