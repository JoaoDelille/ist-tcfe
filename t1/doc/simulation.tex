\section{Operating Point Analysis}
\label{sec:simulation}

To realize the computational simulation of the circuit analysis, ngspice was used. A script was created to perform a DC (Operating Point) analysis, and print the results to this table.\\
It is notable that an extra node, node "e" is present in the table, as well as a short circuited voltage source connecting node n\_0 and "e". This procedure doesn't affect the circuit electrically, but is required for the correct funtioning of ngspice. For the correct simulation of the current dependent voltage source $V_{c}$, such dependent current must be read from a shorted voltage source, as ngspice cannot read it from the resistor $R_{6}$ as \ref{fig:circuit} implies.

This table contains the values obtained, labeled accordingly to the image. The potential differentials in each node, labeled $V_{(1 \; through \; 7)}$ are measured in relation to the common ground, node 0. 

\begin{table}[h]
  \centering
  \begin{tabular}{|l|r|}
    \hline    
    {\bf Name} & {\bf Value [A or V]} \\ \hline
    low & 2.513670e+00\\ \hline
cost & 7.109638e+03\\ \hline
gain & 1.487926e+01\\ \hline
bandwidth & 3.136186e+06\\ \hline
merit & 2.611124e+03\\ \hline

  \end{tabular}
  \caption{Simulated values from Ngspice}
  \label{tab:op}
\end{table}



