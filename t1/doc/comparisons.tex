\section{Comparisons}
\label{sec:comparisons}

As it is possible to evaluate from our results, the theoretical analysis using Octave differs from what was obtained using NGSpice. The currents r1[i] ; r2[i] ; r6[i] ; rf[i] from Table~\ref{tab:op} represent the mesh currents of meshes $m\_1$ , $m\_2$ , $m\_3$ and $m\_4$ respectively. When comparing these to the results obtained with Octave, shown in Table~\ref{tab:mesh}, one finds that they are exactly identical, as would be expected. However the case is not the same for the tension between nodes.\\
The voltages V(1) through V(7) of Table~\ref{tab:op} and of Table~\ref{tab:nodal} both correspond to the tension between the nodes of the same number and the common ground node 0. Despite representing the same tensions, the values obtained in the simulation do not match the ones obtained in Octave.\\
The fact that these values are off by a small margin led us to believe these errors were due to approximations in the numerical methods used by the simulator. In fact, we found that ngspice uses the Newton-Rhaphson method \textbf{ENCONTRAR FONTE DA WIKIPEDIA Q DIZ ISTO} to solve the non linear equations it generates for the OP analysis. This kind of numerical method has an inherent error, meaning that it will give results that are not exact, unlike the method we used for creating and solving linear equations that represent the circuit. The latter method only being limited by the precision of the floating point system used.
