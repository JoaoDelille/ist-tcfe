\section{Theoretical Analysis}
\label{sec:analysis}

The theoretical analysis consists of finding all node voltages and mesh currents using two different methods: Mesh and Nodal Methods. Such values allow us to define the circuit in its entirety.\\

The circuit consists of four elementary meshes. In the circuit, there are four different types of sources: two voltage and two currents sources, which can be divided into independent, $V_a$ and $I_d$, and dependent sources, $V_c$ and $I_b$. There are also seven resistors present.\\
In this analysis, we will be using the Kirchhoffs' laws of current and voltage. The first one mentioned states that current is not acumulated in a node, and therefore the sum of all currents in a node must be zero. The latter one states that the sum of all voltages in any mesh of the circuit must be zero.

\vspace {1cm}
The mesh method consists in assigning a current to each elementary mesh and then solving the Kirchhoffs Voltage Law (KVL) for each of them. The equations use are the following: \[ Va=R1*I1+R4*I1+R3(I1+I2) quad Kb*R3*(I1+I2)=I2 quad R4*I1+(R4+R6+R7)*I3=Kc*I3 quad I4=Id\] \\
One should notice that I3 must be equal to Ic. The potencial diferences between two nodes in any mesh are written using Ohm's law: I*R=V. This allows us to solve the system for the mesh currents. The linear equation system can be represented in matrix form, which allows us to solve it more easly. It is then evident that the first and third equations state that the voltage induced by the source must equal to the sum of the potencial drops in the resistors. The second equation is a manipulation on the dependance condition of the dependant current source present using Ohm's Law to substitute Vb for R3*(I1+I2). The forth one is a direct translation of the fact that the current imposed by the current source is the same one as the one flowing throgh the mesh.

\vspace {1cm}

Colocar matriz correspondente, ao ficheiro de octave de mesh analysis. 

\vspace{1cm}

$$
\begin{bmatrix} 
R_1+R_3+R_4 & R_3       & R_4             & 0 \\       
K_b*R_3     & K_b*R_3-1 & 0               & 0 \\
R_4         &     0     & R_4+R_6+R_7-K_c & 0 \\
0           & 0         & 0               & 1 \\
\end {bmatrix} 
$$
\quad


The nodal method consists in aquiring the relative voltages at each node (one must first define the value of one node as 0) by aquiring as many equations by applying the Kirchhoffs Current Law (KCL) at each node that is not connected to a voltage source. Then, other equations must be aquired in order to get a total equal to the number of nodes. All equations must be linearly independent. By solving this system of equations, we can completly define the system, aquiring branch currents and any other values.\\
There are four nodes that are not connected to any voltage sources. The KCL equations are: \[(V1-V2)/R1+(V33-V2)/R2=(V2-V4)/R3 quad -V6/R6=(V6-V7)/R7 quad (V3-V2)/R2=Kb(V2-V4) quad Id=(V5-V4)/R3+Kb(V2-V4) \]. \\
As aditional equations, we used noted the fact that the diference between two nodes next to a voltage source must be equal to the value given by that source. \[V1=Va quad V7-V4=Kc(-V6/R6)\]\\
An equation was used for the sum of the currents on node 0, in order to get seven equations, and thus, a solvable linear system. \[(V2-V1)/R1+V4/R4+V6/R6=0\]\\
This was also translated into matrix form.

\vspace{1cm}

Falta clocar matriz relacionada com o ficheiro Ocatve 

\vspace {1cm}

Using Octave, once can solve the system. The system was described was Ax=y, where x is equal to the variables in question (currents in the mesh mehtod and voltages in the node method). Using the command y/A', we can solve the system. Note that due to the way Octave works, one must invert the matrix A. Note that the system could also be described as xA'=y, where x is now a line matrix and not a column one.\\
The values obtained were:\\

\vspace{1cm}

$V =$  $R$ x $I$ 
\vspace{1cm} 

As it is possible to evaluate from our results, the theoretical analysis using Octave differs from what was obtained using NGSpice. \textbf{The reason for this is currently unknown, BUT HOPEFULLY WILL BE FIGURED OUT IN A FUTURE COMMIT}
