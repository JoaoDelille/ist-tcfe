\section{Theoretical Analysis}
\label{sec:analysis}

The theoretical analysis consists of finding all node voltages and mesh currents using two different methods: Mesh and Nodal Methods. Such values allow us to define the circuit in its entirety.\\

The circuit consists of four elementary meshes. In the circuit, there are four different types of sources: two voltage and two currents sources, which can be divided into independent, $V_a$ and $I_d$, and dependent sources, $V_c$ and $I_b$. There are also seven resistors present.\\

\vspace {1cm}
The mesh method consists in assigning a current to each elementary mesh and then solving the Kirchhoffs Voltage Law (KVL) for each of them. The equations use are the following: \[ Va=R1*I1+R4*I1+R3(I1+I2) quad \]

\vspace {1cm}

Colocar matriz correspondente, ao ficheiro de octave de mesh analysis. 

\vspace{1cm}

$$
\begin{bmatrix} 
R_1+R_3+R_4 & R_3       & R_4             & 0 \\       
K_b*R_3     & K_b*R_3-1 & 0               & 0 \\
R_4         &     0     & R_4+R_6+R_7-K_c & 0 \\
0           & 0         & 0               & 1 \\
\end {bmatrix} 
$$
\quad


For the Nodal Method, respective equations are constructed for each one, with the idea that the sum of the currents in each one node has to be equal to zero. In that note, it is important to mention that only nodes that aren't connected to voltage sources can be assigned KCL equations. For the other cases additional equations must be created in order to have enough linearly independent equations to discover all the variables. 

\vspace{1cm}

Falta clocar matriz relacionada com o ficheiro Ocatve 

\vspace {1cm}

On both cases the Ohm's Law is applied, in the first method we use the potencial depending of the resistance times the current and in the second one we use the current depending of potencial divided by the resistance of the component: 

\vspace{1cm}

$V =$  $R$ x $I$ 
\vspace{1cm} 

As it is possible to evaluate from our results, the theoretical analysis using Octave differs from what was obtained using NGSpice. \textbf{The reason for this is currently unknown, BUT HOPEFULLY WILL BE FIGURED OUT IN A FUTURE COMMIT}
